% Created 2024-05-30 gio 10:42
% Intended LaTeX compiler: pdflatex
\documentclass[11pt]{article}
\usepackage[utf8]{inputenc}
\usepackage[T1]{fontenc}
\usepackage{graphicx}
\usepackage{longtable}
\usepackage{wrapfig}
\usepackage{rotating}
\usepackage[normalem]{ulem}
\usepackage{amsmath}
\usepackage{amssymb}
\usepackage{capt-of}
\usepackage{hyperref}
\usepackage[scaled]{inter} \renewcommand\familydefault{\sfdefault}
\author{Lorenzo Macelloni}
\date{\today}
\title{titolo}
\hypersetup{
 pdfauthor={Lorenzo Macelloni},
 pdftitle={titolo},
 pdfkeywords={},
 pdfsubject={},
 pdfcreator={Emacs 28.2 (Org mode 9.7)}, 
 pdflang={English}}
\usepackage{calc}
\newlength{\cslhangindent}
\setlength{\cslhangindent}{1.5em}
\newlength{\csllabelsep}
\setlength{\csllabelsep}{0.6em}
\newlength{\csllabelwidth}
\setlength{\csllabelwidth}{0.45em * 3}
\newenvironment{cslbibliography}[2] % 1st arg. is hanging-indent, 2nd entry spacing.
 {% By default, paragraphs are not indented.
  \setlength{\parindent}{0pt}
  % Hanging indent is turned on when first argument is 1.
  \ifodd #1
  \let\oldpar\par
  \def\par{\hangindent=\cslhangindent\oldpar}
  \fi
  % Set entry spacing based on the second argument.
  \setlength{\parskip}{\parskip +  #2\baselineskip}
 }%
 {}
\newcommand{\cslblock}[1]{#1\hfill\break}
\newcommand{\cslleftmargin}[1]{\parbox[t]{\csllabelsep + \csllabelwidth}{#1}}
\newcommand{\cslrightinline}[1]
  {\parbox[t]{\linewidth - \csllabelsep - \csllabelwidth}{#1}\break}
\newcommand{\cslindent}[1]{\hspace{\cslhangindent}#1}
\newcommand{\cslbibitem}[2]
  {\leavevmode\vadjust pre{\hypertarget{citeproc_bib_item_#1}{}}#2}
\makeatletter
\newcommand{\cslcitation}[2]
 {\protect\hyper@linkstart{cite}{citeproc_bib_item_#1}#2\hyper@linkend}
\makeatother\begin{document}

\maketitle
\tableofcontents

\section{Abstract}
\label{sec:org3ed7806}

\section{Introduzione}
\label{sec:org37c25a0}
\subsubsection{Terminologia - sesso, genere, identità di genere, orientamento sessuale}
\label{sec:org612bdef}
Nella discussione di una condizione come l'Incongruenza di Genere (IG) è
opportuno prendere familiarità con una determinata terminologia. Spesso infatti
alcuni termini come ``sesso'' e ``genere'' vengono utilizzati in maniera
interscambiabile nel linguaggio comune, considerando quindi inevitabilmente
correlate quelle che sono le caratteristiche socioculturali e biologiche che
contraddistinguono uomo e donna.

Vediamo quindi quelle che sono le definizioni date dall'NIH per questi concetti
\textsuperscript{\cslcitation{1}{1}} \textsuperscript{\cslcitation{2}{2}}
\begin{enumerate}
\item Sesso Biologico
\label{sec:org4a8dd3a}
Costrutto biologico utilizato per descrivere caratteristiche ormonali, genetiche, anatomiche e biologiche basilari.
Tipicamente è dicotomico, distingue le categorie maschio e femmina.
Tuttavia, esistono anche stati intersessuali, in cui i caratteri ed i cromosomi sessuali non sono definibili all'interno di una sola di queste categorie.
\item Genere (gender)
\label{sec:org1582e17}
Costrutto multidimensionale sociale e culturale; comprende ruoli, attività, comportamenti ed espressioni, che sono tipicamente identificati all'interno di una determinata società come propriamente maschili o femminili.

Sulla base del genere si definisce l'\textbf{identità di genere}.
Questa è la sensazione individuale di sentirsi parte di una determinata categoria di genere, che può essere maschile, femminile, o stati alternativi e non-binari.
L'identità di genere è quindi auto-definita, può cambiare nel corso della vita e non corrisponde necessariamente a quelle che sono le aspettative sociali nei confronti del sesso dell'individuo.

Queste costituiscono infatti il \textbf{ruolo di genere}, ovvero ciò che viene percepito all'interno di una determinata società come proprio di un certo genere, in termini di norme, comportamenti ed espressioni.
L'\textbf{espressione di genere} è quindi la modalità in cui l'individuo decide di mostrare la propra identità di genere attraverso carattersitiche esterne come il proprio abbigliamento, l'utilizzo di determinati pronomni\ldots{}
\item Orientamento Sessuale \textsuperscript{\cslcitation{3}{3}}
\label{sec:org9ab24a9}
L'orientamento sessuale è anch'esso un costrutto complesso che comprende l'attrazione romantica, emotiva e sessuale.
Fa riferimento quindi agli elementi esterni capaci di indurre una risposta nell'individuo.
È quindi definito solitamente sulla base del genere delle persone verso cui un indviduo prova attrazione sessuale o romantica, in rapporto al genere dell'individuo stesso.

I termini più comuni che identificano l'orientamento sessuale di basano su una visione binaria del genere e distinguono quindi persone \emph{eterosessuali}, \emph{omosessuali} e \emph{bisessuali}.
In tempi relativamente recenti sono stati aggiunti termini che prendono in considerazione una visione più moderna e fluida del genere, come \emph{pansessuale} che indica un'attrazione verso gli altri che non dipende dal genere.

L'orientamento sessuale, inoltre, così come il genere, non è binario, ma costituisce uno spettro.
Il primo a prendere in considerazione questa caratteristica fluida della sessualità fu Alfred Kinsey, nel suo libro \emph{Sexual Behavior in the Human Male} del 1948, dove introduce la Scala di Kinsey \textsuperscript{\cslcitation{4}{4}}.
Questa definisce l'orientamento sessuale secondo un gradiente da 0, esclusiva eterossessualità, a 6, esclusiva omosessualità.

L'orientamento sessuale inoltre viene considerato fluido anche nel tempo, infatti questo può cambiare durante la vita di una persona anche in funzione delle circostanze dell'individuo.
\item Transgender and Gender Diverse - TGD
\label{sec:orgbd5e556}
\end{enumerate}
\subsubsection{definizione di incongruenza di genere, cenni storici di medicina di genere}
\label{sec:org0e9e54e}
\begin{enumerate}
\item Varianza` o Non Conformità di Genere
\label{sec:org8825f04}
\item Disforia di Genere
\label{sec:org1c70601}
\item Transessuale
\label{sec:org3532fba}
\item Transgender
\label{sec:orge7a54ce}
\item Intersessuale
\label{sec:orga51bf61}
\end{enumerate}
\subsubsection{Epidemiologia}
\label{sec:org5ef3466}
Nella discussione epidemiologica dei dati che riguardano la popolazione TGD è preferibile evitare i termini ``incidenza'' e ``prevalenza'', questi infatti potrebbero sottointendere in maniera impropria una condizione patologica. Oltretutto, il termine ``incidenza'' non è utilizzabile anche perché sottointende la presenza di un chiaro momento di comparsa dello status TGD, il quale è raramente individuabile.
Si preferiscono quindi i termini ``numero'' e ``proporzione'' per riferirsi alla dimensione assoluta e relativa della popolazione TGD. \textsuperscript{\cslcitation{5}{5}}


Nonostante un'interesse crescente da parte della ricerca nei confronti della salute di questa popolazione, ci sono ancora molti dati epidemiologici anche basilari sui quali si ha poca certezza.
Le stime riportate in vari studi sono infatti fortemente dipendenti dal tipo di metodologia utilizzata per l'indagine e da una diversa definizione del termine transgender.

Diverse pubblicazioni prendono in considerazione solamente coloro che hanno richiesto o intrapreso un percorso chirurgico di riassegnazione del sesso, altri prendono in considerazione le diagnosi di disforia di genere, mentre diversi studi svolti tramite sondaggio nella popolazione generale prendono in considerazione l'autoidentificazione come transgender.
\textsuperscript{\cslcitation{6}{6}}



Prendendo in considerazione i sondaggi condotti nella popolazione che utilizzano definizioni simili i risultati sono consistenti.
Questionari che indagavano nello specifico il termine transgender rilevano una stima che va tra lo 0,3\% e lo 0,5\% tra gli adulti e tra l'1,2\% e il 2,7\% tra bambini ed adolescenti.
Utilizzando una definizione più ampia che include termini come ``incongruenza di genere'' o ``ambivalenza di genere'' la percentuale aumenta a 0,5-4,5\% tra gli adulti e 2,5-8,4\% tra i bambini.
\textsuperscript{\cslcitation{7}{7}}

La dimensione di questa popolazione è inoltre in aumento, su questo concordano sostanzialmente tutte le pubblicazioni che prendono in considerazione l'evoluzione del trend negli anni, indipendentemente da area geografica e modalità di indagine.
\textsuperscript{\cslcitation{8}{8}}

Per quanto riguarda nello specifico la diagnosi clinica di disforia di genere, il DSMV riporta una prevalenza tra il 0,005-0,014\% per le persone AMAB e tra il 0,002\% e 0,003\% per le AFAB, già puntualizzando però come reputi il dato verosimilmente sottostimato.

\textsuperscript{\cslcitation{9}{9}}
Difatti in questo caso la stima prende in considerazione solamente la parte della popolazione TGD che ha ricevuto a tutti gli effetti una diagnosi, per cui appare evidente come questo numero sia sottostimato di diversi ordini di grandezza rispetto ai sondaggi nella popolazione, i quali utilizzano criteri più generici.









\begin{itemize}
\item \url{https://www.publish.csiro.au/sh/sh17067} (zuker → mi sembra di averlo visto citato)
\item \url{https://journals.plos.org/plosone/article?id=10.1371/journal.pone.0299373} (questo me lo ha dato GPT però sembra carino)
\end{itemize}


Per quanto riguarda l'Italia, uno studio del 2023 condotto tramite un sondaggio online diffuso attraverso vari social media, riporta che su 19572 partecipanti il 7,7\% riporta un'identità di genere diversa dal sesso assegnato alla nascita. \textsuperscript{\cslcitation{10}{10}}
Si è anche valutato come i partecipanti TGD avessero un'età media significativamente inferiore rispetto a quelli cisgender.
Inoltre è interessante notare come tra le persone TGD solamente il 41,6\% riportavano un'identita di genere binaria, mentre il 58,4\% si identificavano come non-binari.
\subsubsection{eziologia}
\label{sec:org50521b8}

\subsubsection{diagnosi e criteri diagnostici}
\label{sec:org3e5b082}
\subsubsection{volendo comorbidità psichiatriche?}
\label{sec:orgd4db085}
\subsubsection{clinica}
\label{sec:org58d6d70}
\subsubsection{terapia}
\label{sec:org7ac64e2}
\section{Obiettivi  (?)}
\label{sec:org85ec837}

\section{Materiali e Metodi}
\label{sec:org5de51da}

\section{Risultati}
\label{sec:orgb3425d4}

\section{Discussione}
\label{sec:org4588460}

\section{Conclusioni}
\label{sec:org8d5a2d1}


\section{Bibliografia}
\label{sec:org2f9ea60}

\begin{cslbibliography}{0}{0}
\cslbibitem{1}{\cslleftmargin{1.}\cslrightinline{Sex, Gender, and Sexuality. \textit{National institutes of health (nih)}. Published online August 2022. Accessed May 25, 2024. \url{https://www.nih.gov/nih-style-guide/sex-gender-sexuality}}}

\cslbibitem{2}{\cslleftmargin{2.}\cslrightinline{Sex \& Gender. \textit{Orwhodnihgov}. Accessed May 25, 2024. \url{https://orwh.od.nih.gov/sex-gender}}}

\cslbibitem{3}{\cslleftmargin{3.}\cslrightinline{National Academies of Sciences E, Medicine. \textit{Measuring Sex, Gender Identity, and Sexual Orientation}. (Bates N, Chin M, Becker T, eds.). The National Academies Press; 2022. doi:\href{https://doi.org/10.17226/26424}{10.17226/26424}}}

\cslbibitem{4}{\cslleftmargin{4.}\cslrightinline{Kinsey scale Definition, Meaning, Sexuality, \& Test Britannica. \textit{Wwwbritannicacom}. Published online 2024. Accessed May 25, 2024. \url{https://www.britannica.com/topic/Kinsey-scale}}}

\cslbibitem{5}{\cslleftmargin{5.}\cslrightinline{Coleman E, Radix AE, Bouman WP, et al. Standards of Care for the Health of Transgender and Gender Diverse People, Version 8. \textit{International journal of transgender health}. 2022;23(sup1):S1-S259. doi:\href{https://doi.org/10.1080/26895269.2022.2100644}{10.1080/26895269.2022.2100644}}}

\cslbibitem{6}{\cslleftmargin{6.}\cslrightinline{Collin L, Reisner SL, Tangpricha V, Goodman M. Prevalence of Transgender Depends on the “Case” Definition: A Systematic Review. \textit{The journal of sexual medicine}. 2016;13(4):613-626. doi:\href{https://doi.org/10.1016/j.jsxm.2016.02.001}{10.1016/j.jsxm.2016.02.001}}}

\cslbibitem{7}{\cslleftmargin{7.}\cslrightinline{Zhang Q, Goodman M, Adams N, et al. Epidemiological considerations in transgender health: A systematic review with focus on higher quality data. \textit{International journal of transgender health}. 2020;21(2):125-137. doi:\href{https://doi.org/10.1080/26895269.2020.1753136}{10.1080/26895269.2020.1753136}}}

\cslbibitem{8}{\cslleftmargin{8.}\cslrightinline{Goodman M, Adams N, Corneil T, Kreukels B, Motmans J, Coleman E. Size and Distribution of Transgender and Gender Nonconforming Populations. \textit{Endocrinology and metabolism clinics of north america}. 2019;48(2):303-321. doi:\href{https://doi.org/10.1016/j.ecl.2019.01.001}{10.1016/j.ecl.2019.01.001}}}

\cslbibitem{9}{\cslleftmargin{9.}\cslrightinline{Gender Dysphoria. In: \textit{Diagnostic and Statistical Manual of Mental Disorders}. DSM Library. American Psychiatric Association Publishing; 2022. doi:\href{https://doi.org/10.1176/appi.books.9780890425787.x14_Gender_Dysophoria}{10.1176/appi.books.9780890425787.x14\_Gender\_Dysophoria}}}

\cslbibitem{10}{\cslleftmargin{10.}\cslrightinline{Fisher AD, Marconi M, Castellini G, et al. Estimate and needs of the transgender adult population: the SPoT study. \textit{Journal of endocrinological investigation}. Published online February 2024. doi:\href{https://doi.org/10.1007/s40618-023-02251-9}{10.1007/s40618-023-02251-9}}}

\end{cslbibliography}
\end{document}
