% Created 2024-05-24 ven 19:33
% Intended LaTeX compiler: pdflatex
\documentclass[11pt]{article}
\usepackage[utf8]{inputenc}
\usepackage[T1]{fontenc}
\usepackage{graphicx}
\usepackage{longtable}
\usepackage{wrapfig}
\usepackage{rotating}
\usepackage[normalem]{ulem}
\usepackage{amsmath}
\usepackage{amssymb}
\usepackage{capt-of}
\usepackage{hyperref}
\author{Lorenzo Macelloni}
\date{\today}
\title{titolo}
\hypersetup{
 pdfauthor={Lorenzo Macelloni},
 pdftitle={titolo},
 pdfkeywords={},
 pdfsubject={},
 pdfcreator={Emacs 28.2 (Org mode 9.7)}, 
 pdflang={English}}
\begin{document}

\maketitle
\tableofcontents

\section{Introduzione}
\label{sec:org3966cb0}
\subsubsection{termini - genere, sesso, disforia e incongruenza}
\label{sec:org0853b23}
Nella discussione di una condizione come l'Incongruenza di Genere (IG) è opportuno prendere familiarità con una determinata terminologia.
Spesso infatti alcuni termini come ``sesso'' e ``genere'' vengono utilizzati in maniera interscambiabile nel linguaggio comune, considerando quindi inevitabilmente correlate quelle che sono le caratteristiche socioculturali e biologiche che contraddistinguono uomo e donna.
\begin{enumerate}
\item Sesso Biologico
\label{sec:org6d018ef}
caratteristiche
\item Genere
\label{sec:org752e3ec}

\item Orientamento Sessuale
\label{sec:orgf479089}
\item Varianza o Non Conformità di Genere
\label{sec:org479dddf}
\item Disforia di Genere
\label{sec:orgb6e920e}
\item Transessuale
\label{sec:org0294a8a}
\item Transgender
\label{sec:org4edc4c9}
\item Intersessuale
\label{sec:org41ebcdd}
\end{enumerate}
\subsubsection{definizione di incongruenza di genere}
\label{sec:org13e9171}
\subsubsection{epidemiologia (o simili)}
\label{sec:org87a78c3}
\subsubsection{eziologia}
\label{sec:org3c5db45}
\subsubsection{diagnosi e criteri diagnostici}
\label{sec:orgfbb8e55}
\subsubsection{volendo comorbidità psichiatriche?}
\label{sec:org6414bf7}
\subsubsection{clinica}
\label{sec:orgefefc5d}
\subsubsection{terapia}
\label{sec:orge4bf44e}
\end{document}
