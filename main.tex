% Created 2024-05-25 sab 19:49
% Intended LaTeX compiler: pdflatex
\documentclass[11pt]{article}
\usepackage[utf8]{inputenc}
\usepackage[T1]{fontenc}
\usepackage{graphicx}
\usepackage{longtable}
\usepackage{wrapfig}
\usepackage{rotating}
\usepackage[normalem]{ulem}
\usepackage{amsmath}
\usepackage{amssymb}
\usepackage{capt-of}
\usepackage{hyperref}
\usepackage[scaled]{inter} \renewcommand\familydefault{\sfdefault}
\author{Lorenzo Macelloni}
\date{\today}
\title{titolo}
\hypersetup{
 pdfauthor={Lorenzo Macelloni},
 pdftitle={titolo},
 pdfkeywords={},
 pdfsubject={},
 pdfcreator={Emacs 28.2 (Org mode 9.7)}, 
 pdflang={English}}
\usepackage{calc}
\newlength{\cslhangindent}
\setlength{\cslhangindent}{1.5em}
\newlength{\csllabelsep}
\setlength{\csllabelsep}{0.6em}
\newlength{\csllabelwidth}
\setlength{\csllabelwidth}{0.45em * 2}
\newenvironment{cslbibliography}[2] % 1st arg. is hanging-indent, 2nd entry spacing.
 {% By default, paragraphs are not indented.
  \setlength{\parindent}{0pt}
  % Hanging indent is turned on when first argument is 1.
  \ifodd #1
  \let\oldpar\par
  \def\par{\hangindent=\cslhangindent\oldpar}
  \fi
  % Set entry spacing based on the second argument.
  \setlength{\parskip}{\parskip +  #2\baselineskip}
 }%
 {}
\newcommand{\cslblock}[1]{#1\hfill\break}
\newcommand{\cslleftmargin}[1]{\parbox[t]{\csllabelsep + \csllabelwidth}{#1}}
\newcommand{\cslrightinline}[1]
  {\parbox[t]{\linewidth - \csllabelsep - \csllabelwidth}{#1}\break}
\newcommand{\cslindent}[1]{\hspace{\cslhangindent}#1}
\newcommand{\cslbibitem}[2]
  {\leavevmode\vadjust pre{\hypertarget{citeproc_bib_item_#1}{}}#2}
\makeatletter
\newcommand{\cslcitation}[2]
 {\protect\hyper@linkstart{cite}{citeproc_bib_item_#1}#2\hyper@linkend}
\makeatother\begin{document}

\maketitle
\tableofcontents

\section{Abstract}
\label{sec:org7d0541f}

\section{Introduzione}
\label{sec:orgb103e92}
\subsubsection{termini - genere, sesso, disforia e incongruenza}
\label{sec:orgd200e60}
Nella discussione di una condizione come l'Incongruenza di Genere (IG) è
opportuno prendere familiarità con una determinata terminologia. Spesso infatti
alcuni termini come ``sesso'' e ``genere'' vengono utilizzati in maniera
interscambiabile nel linguaggio comune, considerando quindi inevitabilmente
correlate quelle che sono le caratteristiche socioculturali e biologiche che
contraddistinguono uomo e donna.

Vediamo quindi quelle che sono le definizioni date dall'NIH per questi concetti
\textsuperscript{\cslcitation{1}{1}} \textsuperscript{\cslcitation{2}{2}} (non so se è redundant citare il secondo che è citato dal primo)
\begin{enumerate}
\item Sesso Biologico
\label{sec:orgf2910d6}
Costrutto biologico utilizato per descrivere caratteristiche ormonali, genetiche, anatomiche e biologiche basilari.
Tipicamente è dicotomico, distingue le categorie maschio e femmina.
Tuttavia, esistono anche stati intersessuali, in cui i caratteri ed i cromosomi sessuali non sono definibili all'interno di una sola di queste categorie.
\item Genere (gender)
\label{sec:org8156047}
Costrutto multidimensionale sociale e culturale; comprende ruoli, attività, comportamenti ed espressioni, che sono tipicamente identificati all'interno di una determinata società come propriamente maschili o femminili.

Sulla base del genere si definisce l'\textbf{identità di genere}.
Questa è la sensazione individuale di sentirsi parte di una determinata categoria di genere, che può essere maschile, femminile, o stati alternativi e non-binari.
L'identità di genere è quindi auto-definita, può cambiare nel corso della vita e non corrisponde necessariamente a quelle che sono le aspettative sociali nei confronti del sesso dell'individuo.

Queste costituiscono infatti il \textbf{ruolo di genere}, ovvero ciò che viene percepito all'interno di una determinata società come proprio di un certo genere, in termini di norme, comportamenti ed espressioni.
L'\textbf{espressione di genere} è quindi la modalità in cui l'individuo decide di mostrare la propra identità di genere attraverso carattersitiche esterne come il proprio abbigliamento, l'utilizzo di determinati pronomni\ldots{}
\item Orientamento Sessuale \textsuperscript{\cslcitation{3}{3}}
\label{sec:orgf3aa3c2}
L'orientamento sessuale è anch'esso un costrutto complesso che comprende l'attrazione romantica, emotiva e sessuale.
Fa riferimento quindi agli elementi esterni capaci di indurre una risposta nell'individuo.
È quindi definito solitamente sulla base del genere delle persone verso cui un indviduo prova attrazione sessuale o romantica, in rapporto al genere dell'individuo stesso.

I termini più comuni che identificano l'orientamento sessuale di basano su una visione binaria del genere e distinguono quindi persone \emph{eterosessuali}, \emph{omosessuali} e \emph{bisessuali}.
In tempi relativamente recenti sono stati aggiunti termini che prendono in considerazione una visione più moderna e fluida del genere, come \emph{pansessuale} che indica un'attrazione verso gli altri che non dipende dal genere.

L'orientamento sessuale, inoltre, così come il genere, non è binario, ma costituisce uno spettro.
Il primo a prendere in considerazione questa caratteristica fluida della sessualità fu Alfred Kinsey, nel suo libro \emph{Sexual Behavior in the Human Male} del 1948, dove introduce la Scala di Kinsey \textsuperscript{\cslcitation{4}{4}}.
Questa definisce l'orientamento sessuale secondo un gradiente da 0, esclusiva eterossessualità, a 6, esclusiva omosessualità.

L'orientamento sessuale inoltre viene considerato fluido anche nel tempo, infatti questo può cambiare durante la vita di una persona anche in funzione delle circostanze dell'individuo.
\end{enumerate}
\subsubsection{definizione di incongruenza di genere}
\label{sec:org3dceadf}
\begin{enumerate}
\item Varianza` o Non Conformità di Genere
\label{sec:org77ca7cb}
\item Disforia di Genere
\label{sec:org561c054}
\item Transessuale
\label{sec:org30114b4}
\item Transgender
\label{sec:orgbec14ff}
\item Intersessuale
\label{sec:orgb4c8c03}
\end{enumerate}
\subsubsection{epidemiologia (o simili)}
\label{sec:org9704be3}
\subsubsection{eziologia}
\label{sec:org02538c1}
\subsubsection{diagnosi e criteri diagnostici}
\label{sec:orgd8b70e0}
\subsubsection{volendo comorbidità psichiatriche?}
\label{sec:orgb03143e}
\subsubsection{clinica}
\label{sec:org04b6681}
\subsubsection{terapia}
\label{sec:orgf156321}
\section{Obiettivi  (?)}
\label{sec:orga7b695e}

\section{Materiali e Metodi}
\label{sec:org3e7e19e}

\section{Risultati}
\label{sec:orgb474c13}

\section{Discussione}
\label{sec:orga8b18d6}

\section{Conclusioni}
\label{sec:org11c6bed}


\section{Bibliografia}
\label{sec:org3a98e37}

\begin{cslbibliography}{0}{0}
\cslbibitem{1}{\cslleftmargin{1.}\cslrightinline{Sex, Gender, and Sexuality. \textit{National institutes of health (nih)}. Published online August 2022. Accessed May 25, 2024. \url{https://www.nih.gov/nih-style-guide/sex-gender-sexuality}}}

\cslbibitem{2}{\cslleftmargin{2.}\cslrightinline{Sex \& Gender. \textit{Orwhodnihgov}. Accessed May 25, 2024. \url{https://orwh.od.nih.gov/sex-gender}}}

\cslbibitem{3}{\cslleftmargin{3.}\cslrightinline{National Academies of Sciences E, Medicine. \textit{Measuring Sex, Gender Identity, and Sexual Orientation}. (Bates N, Chin M, Becker T, eds.). The National Academies Press; 2022. doi:\href{https://doi.org/10.17226/26424}{10.17226/26424}}}

\cslbibitem{4}{\cslleftmargin{4.}\cslrightinline{Kinsey scale Definition, Meaning, Sexuality, \& Test Britannica. \textit{Wwwbritannicacom}. Published online 2024. Accessed May 25, 2024. \url{https://www.britannica.com/topic/Kinsey-scale}}}

\end{cslbibliography}
\end{document}
