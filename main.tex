% Created 2024-06-01 sab 16:35
% Intended LaTeX compiler: pdflatex
\documentclass[11pt]{article}
\usepackage[utf8]{inputenc}
\usepackage[T1]{fontenc}
\usepackage{graphicx}
\usepackage{longtable}
\usepackage{wrapfig}
\usepackage{rotating}
\usepackage[normalem]{ulem}
\usepackage{amsmath}
\usepackage{amssymb}
\usepackage{capt-of}
\usepackage{hyperref}
\usepackage[scaled]{inter} \renewcommand\familydefault{\sfdefault}
\author{Lorenzo Macelloni}
\date{\today}
\title{titolo}
\hypersetup{
 pdfauthor={Lorenzo Macelloni},
 pdftitle={titolo},
 pdfkeywords={},
 pdfsubject={},
 pdfcreator={Emacs 28.2 (Org mode 9.7)}, 
 pdflang={English}}
\usepackage{calc}
\newlength{\cslhangindent}
\setlength{\cslhangindent}{1.5em}
\newlength{\csllabelsep}
\setlength{\csllabelsep}{0.6em}
\newlength{\csllabelwidth}
\setlength{\csllabelwidth}{0.45em * 3}
\newenvironment{cslbibliography}[2] % 1st arg. is hanging-indent, 2nd entry spacing.
 {% By default, paragraphs are not indented.
  \setlength{\parindent}{0pt}
  % Hanging indent is turned on when first argument is 1.
  \ifodd #1
  \let\oldpar\par
  \def\par{\hangindent=\cslhangindent\oldpar}
  \fi
  % Set entry spacing based on the second argument.
  \setlength{\parskip}{\parskip +  #2\baselineskip}
 }%
 {}
\newcommand{\cslblock}[1]{#1\hfill\break}
\newcommand{\cslleftmargin}[1]{\parbox[t]{\csllabelsep + \csllabelwidth}{#1}}
\newcommand{\cslrightinline}[1]
  {\parbox[t]{\linewidth - \csllabelsep - \csllabelwidth}{#1}\break}
\newcommand{\cslindent}[1]{\hspace{\cslhangindent}#1}
\newcommand{\cslbibitem}[2]
  {\leavevmode\vadjust pre{\hypertarget{citeproc_bib_item_#1}{}}#2}
\makeatletter
\newcommand{\cslcitation}[2]
 {\protect\hyper@linkstart{cite}{citeproc_bib_item_#1}#2\hyper@linkend}
\makeatother\begin{document}

\maketitle
\tableofcontents

\section{Abstract}
\label{sec:orgfe01153}

\section{Introduzione}
\label{sec:org0a8ec26}
\subsubsection{Terminologia - sesso, genere, identità di genere, orientamento sessuale}
\label{sec:org230ac41}
Nella discussione di una condizione come l'Incongruenza di Genere (IG) è
opportuno prendere familiarità con una determinata terminologia. Spesso infatti
alcuni termini come ``sesso'' e ``genere'' vengono utilizzati in maniera
interscambiabile nel linguaggio comune, considerando quindi inevitabilmente
correlate quelle che sono le caratteristiche socioculturali e biologiche che
contraddistinguono uomo e donna.

Vediamo quindi quelle che sono le definizioni date dall'NIH per questi concetti
\textsuperscript{\cslcitation{1}{1}} \textsuperscript{\cslcitation{2}{2}}
\begin{enumerate}
\item Sesso Biologico
\label{sec:orga8bbfda}
Costrutto biologico utilizato per descrivere caratteristiche ormonali, genetiche, anatomiche e biologiche basilari.
Tipicamente è dicotomico, distingue le categorie maschio e femmina.
Tuttavia, esistono anche stati intersessuali, in cui i caratteri ed i cromosomi sessuali non sono definibili all'interno di una sola di queste categorie.
\item Genere (gender)
\label{sec:org90f6022}
Costrutto multidimensionale sociale e culturale; comprende ruoli, attività, comportamenti ed espressioni, che sono tipicamente identificati all'interno di una determinata società come propriamente maschili o femminili.

Sulla base del genere si definisce l'\textbf{identità di genere}.
Questa è la sensazione individuale di sentirsi parte di una determinata categoria di genere, che può essere maschile, femminile, o stati alternativi e non-binari.
L'identità di genere è quindi auto-definita, può cambiare nel corso della vita e non corrisponde necessariamente a quelle che sono le aspettative sociali nei confronti del sesso dell'individuo.

Queste costituiscono infatti il \textbf{ruolo di genere}, ovvero ciò che viene percepito all'interno di una determinata società come proprio di un certo genere, in termini di norme, comportamenti ed espressioni.
L'\textbf{espressione di genere} è quindi la modalità in cui l'individuo decide di mostrare la propra identità di genere attraverso carattersitiche esterne come il proprio abbigliamento, l'utilizzo di determinati pronomni\ldots{}
\item Orientamento Sessuale \textsuperscript{\cslcitation{3}{3}}
\label{sec:org5228b43}
L'orientamento sessuale è anch'esso un costrutto complesso che comprende l'attrazione romantica, emotiva e sessuale.
Fa riferimento quindi agli elementi esterni capaci di indurre una risposta nell'individuo.
È quindi definito solitamente sulla base del genere delle persone verso cui un indviduo prova attrazione sessuale o romantica, in rapporto al genere dell'individuo stesso.

I termini più comuni che identificano l'orientamento sessuale di basano su una visione binaria del genere e distinguono quindi persone \emph{eterosessuali}, \emph{omosessuali} e \emph{bisessuali}.
In tempi relativamente recenti sono stati aggiunti termini che prendono in considerazione una visione più moderna e fluida del genere, come \emph{pansessuale} che indica un'attrazione verso gli altri che non dipende dal genere.

L'orientamento sessuale, inoltre, così come il genere, non è binario, ma costituisce uno spettro.
Il primo a prendere in considerazione questa caratteristica fluida della sessualità fu Alfred Kinsey, nel suo libro \emph{Sexual Behavior in the Human Male} del 1948, dove introduce la Scala di Kinsey \textsuperscript{\cslcitation{4}{4}}.
Questa definisce l'orientamento sessuale secondo un gradiente da 0, esclusiva eterossessualità, a 6, esclusiva omosessualità.

L'orientamento sessuale inoltre viene considerato fluido anche nel tempo, infatti questo può cambiare durante la vita di una persona anche in funzione delle circostanze dell'individuo.
\item Transgender and Gender Diverse - TGD
\label{sec:orgeed431f}
\end{enumerate}
\subsubsection{definizione di incongruenza di genere, cenni storici di medicina di genere}
\label{sec:org1938fe0}
\begin{enumerate}
\item Varianza` o Non Conformità di Genere
\label{sec:org495efe2}
\item Disforia di Genere
\label{sec:orgd07b57d}
\item Transessuale
\label{sec:orgbfd7148}
\item Transgender
\label{sec:org3ff1f93}
\item Intersessuale
\label{sec:orgfa10150}
\end{enumerate}
\subsubsection{Epidemiologia}
\label{sec:org3f0bc30}
Nella discussione epidemiologica dei dati che riguardano la popolazione TGD è preferibile evitare i termini ``incidenza'' e ``prevalenza'', questi infatti potrebbero sottointendere in maniera impropria una condizione patologica. Oltretutto, il termine ``incidenza'' non è utilizzabile anche perché sottointende la presenza di un chiaro momento di comparsa dello status TGD, il quale è raramente individuabile.
Si preferiscono quindi i termini ``numero'' e ``proporzione'', per riferirsi alla dimensione assoluta e relativa della popolazione TGD. \textsuperscript{\cslcitation{5}{5}}


Nonostante un'interesse crescente da parte della ricerca nei confronti della salute di questa popolazione, ci sono ancora molti dati epidemiologici anche basilari sui quali si ha poca certezza.
Le stime riportate in vari studi sono infatti fortemente dipendenti dal tipo di metodologia utilizzata per l'indagine e dalla definizione data del termine transgender.
A seconda delle pubblicazioni vengono presi in considerazione certe volte solamente color che hanno richiesto o intrapreso un percorso chirurgico di riassegnazione del sesso, altri prendono in considerazione le diagnosi di disforia di genere, mentre diversi studi svolti tramite sondaggio nella popolazione generale prendono in considerazione l'autoidentificazione come transgender.
\textsuperscript{\cslcitation{6}{6}}

Per quanto riguarda nello specifico la diagnosi clinica di disforia di genere, il DSMV riporta una prevalenza tra il 0,005-0,014\% per le persone AMAB e tra il 0,002\% e 0,003\% per le AFAB, già puntualizzando però come reputi il dato verosimilmente sottostimato.

\textsuperscript{\cslcitation{7}{7}}
Questa stima infatti prende in considerazione solamente la parte della popolazione TGD che ha ricevuto a tutti gli effetti una diagnosi, per cui appare evidente come questo numero sia sottostimato di diversi ordini di grandezza rispetto ai sondaggi nella popolazione, i quali utilizzano criteri più generici.

Prendendo in considerazione i sondaggi condotti nella popolazione che utilizzano definizioni simili, i risultati sono consistenti.
Questionari che indagavano nello specifico il termine ``transgender'' rilevavano una stima che va tra lo 0,3\% e lo 0,5\% tra gli adulti e tra l'1,2\% e il 2,7\% tra bambini ed adolescenti.
Utilizzando una definizione più ampia che include termini come ``incongruenza di genere'' o ``ambivalenza di genere'' la percentuale aumenta a 0,5-4,5\% tra gli adulti e 2,5-8,4\% nella popolazione adolescente e pediatrica.
\textsuperscript{\cslcitation{8}{8}}

La dimensione di questa popolazione è inoltre in aumento, su questo concordano sostanzialmente tutte le pubblicazioni che prendono in considerazione l'evoluzione del trend negli anni, indipendentemente da area geografica e modalità di indagine.
\textsuperscript{\cslcitation{9}{9}}


Per quanto riguarda l'Italia, uno studio del 2023 condotto tramite un sondaggio online diffuso attraverso vari social media, riporta che su 19572 partecipanti il 7,7\% riporta un'identità di genere diversa dal sesso assegnato alla nascita. \textsuperscript{\cslcitation{10}{10}}
Si è anche valutato come i partecipanti TGD avessero un'età media significativamente inferiore rispetto a quelli cisgender.
Inoltre è interessante notare come tra le persone TGD solamente il 41,6\% riportavano un'identita di genere binaria, mentre il 58,4\% si identificavano come non-binari.
\subsubsection{Eziologia}
\label{sec:org5c58e55}

Attualmente non sono ancora stati identificati dei chiari fattori eziologici determinanti nell'insorgenza di una incongruenza di genere.
Come molte altre patologie, l'ipotesi più attuale comprende l'interazione tra molteplici fattori di tipo biologico, genetico e psicosociale.
\begin{enumerate}
\item Fattori Neurologici
\label{sec:org74cd7b7}
Il coinvolgimento neurologico si basa sull'ipotesi che i soggetti transgender abbiano delle differenze nello sviluppo dei circuiti cerebrali, rispetto ai cisgender, e che questo sia determinante nell'insorgenza dell'incongruenza di genere.
La base biologica di questa teoria è la differenza già nota tra cervello maschile e femminile nei soggetti cisgender; questa si presenta sia in un leggero vantaggio dell'uno o l'altro sesso in alcuni task cognitivi, sia in una vera e propria differenza anatomica di trofismo di alcune zone cerebrali piuttosto che altre. \textsuperscript{\cslcitation{11}{11}}

Sono diversi i fattori che intervengono nel determinare queste differenze e non tutti sono conosciuti; sicuramente è presente un'influenza ambientale, com'è reso evidente dal fatto che queste differenze tra maschi e femmine sono diverse in diverse aree geografiche, è molto probabile anche un ruolo degli ormoni sessuali durante sviluppo, difatti le differenze di trofismo sono state associate ad aree con diversa quantità di recettori estrogenici e androgenici nelle varie aree cerebrali. \textsuperscript{\cslcitation{12}{12}}

Per quanto riguarda la popolazione TGD, seppur siano state dimostrate alcune differenze strutturali e funzionali nel cervello degli individui TGD, non è ancora stato indiviuato in letteratura un pattern preciso che si possa associare chiaramente a determinati cambiamenti strutturali.
Alcuni studi dimostrano come la morfologia cervello di individui con incongruenza di genere sia complessivamente più simile ad individui cisgender del sesso assegnato alla nascita rispetto a individui cisgender dell'identità di genere scelta \textsuperscript{\cslcitation{13}{13}}.
Tuttavia esiste anche evidenza discordante, ad esempio gli studi riguardanti la struttura della materia bianca tedono a concordare sull'esistenza di un fenotipo intermedio negli individui transgender, differente da quello di entrambi maschi e femmine cisgender\textsuperscript{\cslcitation{14}{14}} \textsuperscript{\cslcitation{15}{15}}\textsuperscript{\cslcitation{16}{16}}.

Complessivamente è difficile giungere a conclusioni chiare, gli studi infatti sono limitati dall'uso di metodiche di imaging non invasive e popolazioni di piccole dimenioni; oltretutto molti prendono in considerazione sia l'identità di genere che l'orientamento sessuale, rendendo difficile differenziare chiaramente l'influenza delle due variabili.
\item Fattori Genetici
\label{sec:org4ea8d80}
Diversi studi ipotizzano la presenza di una componente genetica nella costruzione dell'identità di genere quindi dell'incongruenza, tuttavia al momento non sono stati trovati geni specifici direttamente coinvolti.

Diversi studi sono stati condotti su gemelli monozigoti, mettendo in evidenaza come questi abbiano un tasso di concordanza maggiore sia per quanto riguarda l'identità sia per l'incongruenza di genere. \textsuperscript{\cslcitation{17}{17}} \textsuperscript{\cslcitation{18}{18}}

Uno studio ha studiato invece il potenziale ruolo dei geni coinvolti nel \emph{signaling} degli ormoni sessuali, mettendo in evidenza come alcune varianti genetiche siano correlate all'incongruenza di genere in alcuni pazienti AMAB, facendo anche valutazioni ed ipotesi sul meccanismo di azione degli specifici polimorfismi. \textsuperscript{\cslcitation{19}{19}}
\item Fattori Endocrini
\label{sec:orgbab97d2}
???
\item Fattori Psicologici e Sociali
\label{sec:orgb501234}

La maggior parte degli studi prendono in considerazione il probabile intervento di vari fattori psicologici nella genesi dell'identità di genere e quindi dell'incongruenza, diverse teorie psicologiche identificano elementi differenti che potrebbero agire in diverse fasi della vita dell'individuo.

La teoria più primitiva è quella \emph{psicodinamica}, che si rifa addirittura alla teoria Freudiana dell'identificazione, ipotizzando un intervento importante dell'esperienza infantile nella determinazione dell'identità di genere. Secondo la ``teoria dell'identificazione'' di Freud il bambino tende ad identificarsi con il genitore del sesso opposto in quello che è il complesso di Edipo. \textsuperscript{\cslcitation{20}{20}}
Nonostante ancora non venga espresso chiaramente il concetto di incongruenza di genere, è chiaro come questi concetti sottointendano l'ipotesi di una certa variabilità e fluidità nell'identità di genere.

La successiva teoria \emph{psicosociale} prende invece in considerazione l'intervento di fattori ambientali e culturali nella determinazione dell'identità di genere.

L'apprendimento del comportamento e ruolo di genere avvengono tramite l'osservazione e l'imitazione nel contesto sociale, andando a replicare quelli che sono i comportamenti considerati adeguati al proprio sesso, in un procedimento graduale che si sviluppa negli anni \textsuperscript{\cslcitation{21}{21}}.
Nella popolazione TGD verosimilmente lo sviluppo della propria identità di genere avviene in modo anologo, con simili fattori determinanti,\textsuperscript{\cslcitation{22}{22}} tuttavia, uno studio sulla popolazione pediatrica ha individuato un possibile rallentamento nell'acquisizione di un'identità di genere stabile in bambini che riconoscono un'identità di genere non corrispondente al proprio sesso biologico\textsuperscript{\cslcitation{23}{23}}.
Un ruolo importante è anche attribuito alla presenza nell'ambiente familiare e sociale di una pressione sul bambino a comportarsi in maniera conforme al proprio sesso biologico. Questa insistenza si riflette negativamente sull'adattamento psicologico del bambino, motivo per cui si reputa al contrario ottimale un ambiente in cui il bambino possa sentirsi libero sia di confermare la propria appartenenza al sesso biologico, sia gli venga data la possibilità di esplorare identità di genere alternative. \textsuperscript{\cslcitation{24}{24}}


% nella tesi di lorenzo auricchio c'è una parte sul trauma però io non ho trovato niente di articoli, nemmeno quelli citati da lui ne parlano (non dicono quello che c'è scritto...)
\end{enumerate}
\subsubsection{diagnosi e criteri diagnostici}
\label{sec:orgff59186}
\subsubsection{clinica}
\label{sec:org53447c0}
\subsubsection{terapia}
\label{sec:org7dcc746}
\subsubsection{BIA ??}
\label{sec:orgb7531b2}
\section{Obiettivi  (?)}
\label{sec:org69a1c51}

\section{Materiali e Metodi}
\label{sec:orgc2b8fde}

\section{Risultati}
\label{sec:org7dc7428}

\section{Discussione}
\label{sec:org3c6e88a}

\section{Conclusioni}
\label{sec:orge48eb6c}


\section{Bibliografia}
\label{sec:orga34a4f5}

\begin{cslbibliography}{0}{0}
\cslbibitem{1}{\cslleftmargin{1.}\cslrightinline{Sex, Gender, and Sexuality. \textit{National institutes of health (nih)}. Published online August 2022. Accessed May 25, 2024. \url{https://www.nih.gov/nih-style-guide/sex-gender-sexuality}}}

\cslbibitem{2}{\cslleftmargin{2.}\cslrightinline{Sex \& Gender. \textit{Orwhodnihgov}. Accessed May 25, 2024. \url{https://orwh.od.nih.gov/sex-gender}}}

\cslbibitem{3}{\cslleftmargin{3.}\cslrightinline{National Academies of Sciences E, Medicine. \textit{Measuring Sex, Gender Identity, and Sexual Orientation}. (Bates N, Chin M, Becker T, eds.). The National Academies Press; 2022. doi:\href{https://doi.org/10.17226/26424}{10.17226/26424}}}

\cslbibitem{4}{\cslleftmargin{4.}\cslrightinline{Kinsey scale Definition, Meaning, Sexuality, \& Test Britannica. \textit{Wwwbritannicacom}. Published online 2024. Accessed May 25, 2024. \url{https://www.britannica.com/topic/Kinsey-scale}}}

\cslbibitem{5}{\cslleftmargin{5.}\cslrightinline{Coleman E, Radix AE, Bouman WP, et al. Standards of Care for the Health of Transgender and Gender Diverse People, Version 8. \textit{International journal of transgender health}. 2022;23(sup1):S1-S259. doi:\href{https://doi.org/10.1080/26895269.2022.2100644}{10.1080/26895269.2022.2100644}}}

\cslbibitem{6}{\cslleftmargin{6.}\cslrightinline{Collin L, Reisner SL, Tangpricha V, Goodman M. Prevalence of Transgender Depends on the “Case” Definition: A Systematic Review. \textit{The journal of sexual medicine}. 2016;13(4):613-626. doi:\href{https://doi.org/10.1016/j.jsxm.2016.02.001}{10.1016/j.jsxm.2016.02.001}}}

\cslbibitem{7}{\cslleftmargin{7.}\cslrightinline{Gender Dysphoria. In: \textit{Diagnostic and Statistical Manual of Mental Disorders}. DSM Library. American Psychiatric Association Publishing; 2022. doi:\href{https://doi.org/10.1176/appi.books.9780890425787.x14_Gender_Dysophoria}{10.1176/appi.books.9780890425787.x14\_Gender\_Dysophoria}}}

\cslbibitem{8}{\cslleftmargin{8.}\cslrightinline{Zhang Q, Goodman M, Adams N, et al. Epidemiological considerations in transgender health: A systematic review with focus on higher quality data. \textit{International journal of transgender health}. 2020;21(2):125-137. doi:\href{https://doi.org/10.1080/26895269.2020.1753136}{10.1080/26895269.2020.1753136}}}

\cslbibitem{9}{\cslleftmargin{9.}\cslrightinline{Goodman M, Adams N, Corneil T, Kreukels B, Motmans J, Coleman E. Size and Distribution of Transgender and Gender Nonconforming Populations. \textit{Endocrinology and metabolism clinics of north america}. 2019;48(2):303-321. doi:\href{https://doi.org/10.1016/j.ecl.2019.01.001}{10.1016/j.ecl.2019.01.001}}}

\cslbibitem{10}{\cslleftmargin{10.}\cslrightinline{Fisher AD, Marconi M, Castellini G, et al. Estimate and needs of the transgender adult population: the SPoT study. \textit{Journal of endocrinological investigation}. Published online February 2024. doi:\href{https://doi.org/10.1007/s40618-023-02251-9}{10.1007/s40618-023-02251-9}}}

\cslbibitem{11}{\cslleftmargin{11.}\cslrightinline{Miller DI, Halpern DF. The new science of cognitive sex differences. \textit{Trends in cognitive sciences}. 2014;18(1):37-45. doi:\href{https://doi.org/10.1016/j.tics.2013.10.011}{10.1016/j.tics.2013.10.011}}}

\cslbibitem{12}{\cslleftmargin{12.}\cslrightinline{Goldstein JM. Normal Sexual Dimorphism of the Adult Human Brain Assessed by In Vivo Magnetic Resonance Imaging. \textit{Cerebral cortex}. 2001;11(6):490-497. doi:\href{https://doi.org/10.1093/cercor/11.6.490}{10.1093/cercor/11.6.490}}}

\cslbibitem{13}{\cslleftmargin{13.}\cslrightinline{Frigerio A, Ballerini L, Valdés Hernández M. Structural, Functional, and Metabolic Brain Differences as a Function of Gender Identity or Sexual Orientation: A Systematic Review of the Human Neuroimaging Literature. \textit{Archives of sexual behavior}. 2021;50(8):3329-3352. doi:\href{https://doi.org/10.1007/s10508-021-02005-9}{10.1007/s10508-021-02005-9}}}

\cslbibitem{14}{\cslleftmargin{14.}\cslrightinline{Mueller SC, De Cuypere G, T’Sjoen G. Transgender Research in the 21st Century: A Selective Critical Review From a Neurocognitive Perspective. \textit{American journal of psychiatry}. 2017;174(12):1155-1162. doi:\href{https://doi.org/10.1176/appi.ajp.2017.17060626}{10.1176/appi.ajp.2017.17060626}}}

\cslbibitem{15}{\cslleftmargin{15.}\cslrightinline{Guillamon A, Junque C, Gómez-Gil E. A Review of the Status of Brain Structure Research in Transsexualism. \textit{Archives of sexual behavior}. 2016;45(7):1615-1648. doi:\href{https://doi.org/10.1007/s10508-016-0768-5}{10.1007/s10508-016-0768-5}}}

\cslbibitem{16}{\cslleftmargin{16.}\cslrightinline{Manzouri A, Savic I. Possible Neurobiological Underpinnings of Homosexuality and Gender Dysphoria. \textit{Cerebral cortex}. 2019;29(5):2084-2101. doi:\href{https://doi.org/10.1093/cercor/bhy090}{10.1093/cercor/bhy090}}}

\cslbibitem{17}{\cslleftmargin{17.}\cslrightinline{Kauffman RP, Guerra C, Thompson CM, Stark A. Concordance for Gender Dysphoria in Genetic Female Monozygotic (Identical) Triplets. \textit{Archives of sexual behavior}. 2022;51(7):3647-3651. doi:\href{https://doi.org/10.1007/s10508-022-02409-1}{10.1007/s10508-022-02409-1}}}

\cslbibitem{18}{\cslleftmargin{18.}\cslrightinline{Diamond M. Transsexuality Among Twins: Identity Concordance, Transition, Rearing, and Orientation. \textit{International journal of transgenderism}. 2013;14(1):24-38. doi:\href{https://doi.org/10.1080/15532739.2013.750222}{10.1080/15532739.2013.750222}}}

\cslbibitem{19}{\cslleftmargin{19.}\cslrightinline{Foreman M, Hare L, York K, et al. Genetic Link Between Gender Dysphoria and Sex Hormone Signaling. \textit{The journal of clinical endocrinology \& metabolism}. 2019;104(2):390-396. doi:\href{https://doi.org/10.1210/jc.2018-01105}{10.1210/jc.2018-01105}}}

\cslbibitem{20}{\cslleftmargin{20.}\cslrightinline{Benjamin J. Father and daughter: Identification with difference—a contribution to gender heterodoxy. \textit{Psychoanalytic dialogues}. 1991;1(3):277-299. doi:\href{https://doi.org/10.1080/10481889109538900}{10.1080/10481889109538900}}}

\cslbibitem{21}{\cslleftmargin{21.}\cslrightinline{Steensma TD, Kreukels BP, de Vries AL, Cohen-Kettenis PT. Gender identity development in adolescence. \textit{Hormones and behavior}. 2013;64(2):288-297. doi:\href{https://doi.org/10.1016/j.yhbeh.2013.02.020}{10.1016/j.yhbeh.2013.02.020}}}

\cslbibitem{22}{\cslleftmargin{22.}\cslrightinline{Mehrtens I, Addante S. Transgender and Gender Diverse Identity Development in Pediatric Populations. \textit{Pediatric annals}. 2023;52(12). doi:\href{https://doi.org/10.3928/19382359-20231016-05}{10.3928/19382359-20231016-05}}}

\cslbibitem{23}{\cslleftmargin{23.}\cslrightinline{Zucker KJ, Bradley SJ, Kuksis M, et al. Gender Constancy Judgments in Children with Gender Identity Disorder: Evidence for a Developmental Lag. \textit{Archives of sexual behavior}. 1999;28(6):475-502. doi:\href{https://doi.org/10.1023/A:1018713115866}{10.1023/A:1018713115866}}}

\cslbibitem{24}{\cslleftmargin{24.}\cslrightinline{Egan SK, Perry DG. Gender identity: A multidimensional analysis with implications for psychosocial adjustment. \textit{Developmental psychology}. 2001;37(4):451-463. doi:\href{https://doi.org/10.1037/0012-1649.37.4.451}{10.1037/0012-1649.37.4.451}}}

\end{cslbibliography}
\end{document}
