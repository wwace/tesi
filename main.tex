% Created 2024-06-06 gio 18:26
% Intended LaTeX compiler: pdflatex
\documentclass[11pt]{article}
\usepackage[utf8]{inputenc}
\usepackage[T1]{fontenc}
\usepackage{graphicx}
\usepackage{longtable}
\usepackage{wrapfig}
\usepackage{rotating}
\usepackage[normalem]{ulem}
\usepackage{amsmath}
\usepackage{amssymb}
\usepackage{capt-of}
\usepackage{hyperref}
\usepackage[scaled]{inter} \renewcommand\familydefault{\sfdefault}
\usepackage{setspace} \onehalfspacing
\usepackage{geometry} \geometry{a4paper, top=2.5cm, bottom=2.5cm, left=3.5cm, right=2.5cm }
\author{Lorenzo Macelloni}
\date{\today}
\title{titolo}
\hypersetup{
 pdfauthor={Lorenzo Macelloni},
 pdftitle={titolo},
 pdfkeywords={},
 pdfsubject={},
 pdfcreator={Emacs 28.2 (Org mode 9.7)}, 
 pdflang={English}}
\usepackage{calc}
\newlength{\cslhangindent}
\setlength{\cslhangindent}{1.5em}
\newlength{\csllabelsep}
\setlength{\csllabelsep}{0.6em}
\newlength{\csllabelwidth}
\setlength{\csllabelwidth}{0.45em * 3}
\newenvironment{cslbibliography}[2] % 1st arg. is hanging-indent, 2nd entry spacing.
 {% By default, paragraphs are not indented.
  \setlength{\parindent}{0pt}
  % Hanging indent is turned on when first argument is 1.
  \ifodd #1
  \let\oldpar\par
  \def\par{\hangindent=\cslhangindent\oldpar}
  \fi
  % Set entry spacing based on the second argument.
  \setlength{\parskip}{\parskip +  #2\baselineskip}
 }%
 {}
\newcommand{\cslblock}[1]{#1\hfill\break}
\newcommand{\cslleftmargin}[1]{\parbox[t]{\csllabelsep + \csllabelwidth}{#1}}
\newcommand{\cslrightinline}[1]
  {\parbox[t]{\linewidth - \csllabelsep - \csllabelwidth}{#1}\break}
\newcommand{\cslindent}[1]{\hspace{\cslhangindent}#1}
\newcommand{\cslbibitem}[2]
  {\leavevmode\vadjust pre{\hypertarget{citeproc_bib_item_#1}{}}#2}
\makeatletter
\newcommand{\cslcitation}[2]
 {\protect\hyper@linkstart{cite}{citeproc_bib_item_#1}#2\hyper@linkend}
\makeatother\begin{document}

\maketitle
\tableofcontents

\section{Valutazione e approccio ad un individuo con incongruenza di genere}
\label{sec:orgadee669}

La gestione di un individuo TGD non è compito semplice per il clinico, per questo motivo la \emph{World Professional Association for Transgender Health} (WPATH) stila un documento per stabilire quelle che sono le migliori pratiche cliniche da mettere in atto, questo è lo /Standards of Care of Transgender and Gender Diverse People/(SOC). \textsuperscript{1}
La WPATH è un'organizzazione non-profit interdisciplinare professionale ed educativa, il cui scopo è quello di promuovere un alto standard di cura per tutta la popolazione TGD. \textsuperscript{2}.
Gli SOC rappresentano un insieme di linee guida riconosciute a livello internazionale per la presa in carico di individui TGD, con l'obiettivo di portarli a raggiungere una situazione di salute a livello fisico e psicologico, l'ultima edizione pubblicata sono gli SOC-8 del 2022. \\

Queste raccomandazioni non sono pensate esclusivamente per i professionisti sanitari, difatti un intero capitolo è dedicato all'educazione per la popolazione generale, punto fondamentale per combattere contro la discriminazione ancora molto diffusa nei confronti degli individui TGD.
Atti di discrimniatori, di intolleranza e violenza nei confronti della popolazione TGD rappresentanto un fenomeno frequente, che impatta in modo importante la salute e la sicurezza di questi individui, con una percentuale di violenza riportata che arriva fino all'80\% in alcune indagini\textsuperscript{3}.

Anche per quanto riguarda il personale sanitario, le competenze risultano spesso insufficienti, specialmente nel personale non specializzato \textsuperscript{4}, con una buona percentuale di persone TGD che riportano esplicitamente di evitare per quanto possibile l'utilizzo dei servizi sanitari per paura di essere discriminati o subire maltrattamenti \textsuperscript{5}. Questo risulta estremamente problematico, andando ad limitare e rendere più difficile l'accesso a terapie importanti di affermazione di genere e rendendo più difficoltosa la gestione di una condizione già intrinsecamente complessa\textsuperscript{6}.


Gli SOC individuano un diverso approccio all'individuo TGD secondo l'età, esistono infatti linee guida separate per adulti, adolescenti e bambini.

Nell'adulto, il primo compito del professionista sanitario è di effettuare una corretta valutazione della presenza di incongruenza di genere e di identificare altre eventuali problematiche psichiatriche \textsuperscript{1}.
Successivamente è importante informare ed educare l'adulto TGD per quanto riguarda quelli che sono i possibili percorsi di affermazione di genere, sia medici che chirurgici, dato che è stato dimostrato da vari studi come questi abbiano un impatto positivo importante sulla salute mentale nei soggetti TGD\textsuperscript{7}, migliorando la qualità della vita, diminuendo i sintomi di ansia e depressione\textsuperscript{8} e il rischio suicidario\textsuperscript{9}.

La decisione di intraprendere un percorso di affermazione di genere è un passo importante per l'individuo TGD ed una decisione che spesso viene presa in collaborazione con un profesionista sanitario \textsuperscript{1}, anche se in alcuni casi, solamente per le terapie ormonali, vengono utilizzati con successo dei modelli che prediligono la decisione dell'adulto TGD, tipicamente chiamati modelli a ``consenso informato'' \textsuperscript{10}\textsuperscript{11}.
In ogni caso è fondamentale assicurasi che il soggetto sia in grado di comprendere quali sono rischi e benefici del trattamento per essere in grado di dare il suo consenso \textsuperscript{1}, escludendo malattie mentali che possono interferire, in particolar modo sintomi di decadimento cognitivo o psicotici\textsuperscript{12}.

Un'altra parte importante del percorso di un individuo TGD è quella di transizione sociale, che può dare grande beneficio al soggetto, migliorandone la salute mentale e la qualità della vita\textsuperscript{1}.
Tuttavia, esistono anche circostanze in cui l'individuo non desidera effettuare la transizione sociale per varie motivazioni, spesso correlate ad una paura di abbandono da parte di amici e familiari,
\end{document}
