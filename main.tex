% Created 2024-06-03 lun 17:39
% Intended LaTeX compiler: pdflatex
\documentclass[11pt]{article}
\usepackage[utf8]{inputenc}
\usepackage[T1]{fontenc}
\usepackage{graphicx}
\usepackage{longtable}
\usepackage{wrapfig}
\usepackage{rotating}
\usepackage[normalem]{ulem}
\usepackage{amsmath}
\usepackage{amssymb}
\usepackage{capt-of}
\usepackage{hyperref}
\usepackage[scaled]{inter} \renewcommand\familydefault{\sfdefault}
\usepackage{setspace} \onehalfspacing
\usepackage{geometry} \geometry{a4paper, top=2.5cm, bottom=2.5cm, left=3.5cm, right=2.5cm }
\author{Lorenzo Macelloni}
\date{\today}
\title{titolo}
\hypersetup{
 pdfauthor={Lorenzo Macelloni},
 pdftitle={titolo},
 pdfkeywords={},
 pdfsubject={},
 pdfcreator={Emacs 28.2 (Org mode 9.7)}, 
 pdflang={English}}
\usepackage{calc}
\newlength{\cslhangindent}
\setlength{\cslhangindent}{1.5em}
\newlength{\csllabelsep}
\setlength{\csllabelsep}{0.6em}
\newlength{\csllabelwidth}
\setlength{\csllabelwidth}{0.45em * 2}
\newenvironment{cslbibliography}[2] % 1st arg. is hanging-indent, 2nd entry spacing.
 {% By default, paragraphs are not indented.
  \setlength{\parindent}{0pt}
  % Hanging indent is turned on when first argument is 1.
  \ifodd #1
  \let\oldpar\par
  \def\par{\hangindent=\cslhangindent\oldpar}
  \fi
  % Set entry spacing based on the second argument.
  \setlength{\parskip}{\parskip +  #2\baselineskip}
 }%
 {}
\newcommand{\cslblock}[1]{#1\hfill\break}
\newcommand{\cslleftmargin}[1]{\parbox[t]{\csllabelsep + \csllabelwidth}{#1}}
\newcommand{\cslrightinline}[1]
  {\parbox[t]{\linewidth - \csllabelsep - \csllabelwidth}{#1}\break}
\newcommand{\cslindent}[1]{\hspace{\cslhangindent}#1}
\newcommand{\cslbibitem}[2]
  {\leavevmode\vadjust pre{\hypertarget{citeproc_bib_item_#1}{}}#2}
\makeatletter
\newcommand{\cslcitation}[2]
 {\protect\hyper@linkstart{cite}{citeproc_bib_item_#1}#2\hyper@linkend}
\makeatother\begin{document}

\maketitle
\tableofcontents

\section{Valutazione e trattamento di un individuo con incongruenza di genere}
\label{sec:org2c37482}

La gestione di un individuo TGD non è compito semplice per il clinico, per questo motivo la \emph{World Professional Association for Transgender Health} (WPATH) stila un insieme di linee guida contenute nello /Standards of Care of Transgender and Gender Diverse People/(SOC). \textsuperscript{1}
La WPATH è un'organizzazione non-profit interdisciplinare pofessionale ed educativa, il cui scopo è quello di promuovere un'alto standard di cura per tutta la popolazione TGD. \textsuperscript{2}.
L'SOC è un insieme di linee guida riconosciute a livello internazionale per la presa in carico di individui TGD, con l'obeittivo di portarli a raggiungere una situazione di salute a livello fisico e psicologico, l'ultima edizione pubblicata è l'SOC-8 del 2022.

Queste raccomandazioni non sono pensate esclusivamente per i professionisti sanitari, difatti un intero capitolo è dedicato all'educazione per la popolazione generale.
\end{document}
